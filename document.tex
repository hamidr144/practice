\documentclass[14pt,a4paper]{article}
%\renewcommand{\baselinestretch}{1.5}
\linespread{1.3}
\usepackage[T2A]{fontenc} % Поддержка русских букв
\usepackage[utf8]{inputenc} 
\usepackage[english,russian]{babel}  


 
\begin{document}
\thispagestyle{empty}

\begin{center}
\textbf{САНКТ-ПЕТЕРБУРГСКИЙ НАЦИОНАЛЬНЫЙ ИССЛЕДОВАТЕЛЬСКИЙ УНИВЕРСИТЕТ ИНФОРМАЦИОННЫХ ТЕХНОЛОГИЙ, 
МЕХАНИКИ И ОПТИКИ}
\end{center}


\newpage
\tableofcontents
\newpage
\section{Система компьютерной верстки \TeX(\LaTeX)}
\subsection{\TeX}
\subsubsection{История \TeX}
\TeX{} ~--- система компьютерной верстки, разработанная Дональдом Кнутом, которая
предназанчена для компьютерной верстки текста и математических формул. Кнут
начал разрабатывать систему в 1977 году, и первая версия \TeX{} вышла 1979
года. В 1982 году вышла заново переписанная версия \TeX'а, которой было дано
название TeX82. И с версии \TeX{} 3.0, которая получила лучшую поддержку
8-битных символов и различных языков, используеться нумирация: каждое обновление
добавляет в конец номера версии десятичную цифру так, что бы она приблежалась к
числу \begin{math} \pi \end{math}.

\subsubsection{Особенности \TeX}
В \TeX{} пользователь пишет тескст и задает лишь струкуту самого текста, а система
сама формирует документ на основе выбранного шаблона. Для задания структуры
используеться собственный язык разметки \TeX'а, все это содержиться в фалье с
расширением .tor, и \TeX{} транслирует в файл .dvi.

\TeX{} можно использовать для создания разных видов докуметов: книги, статьи,
отчеты, письма и др.

\subsection{\LaTeX}
\LaTeX{} ~--- макропакет компьютерной верстки \TeX{}. Он не добаляет возможности
в \TeX{}, а лишь позволяет автоматизировать задачи набора текста(умерация
разделов и формул, перекресные ссылки, размещение таблиц и т. п.). Первую версию
выпустил Лесли Лэмпортв 1984 году. В 1994 году была выпущена вторая версия
\LaTeX -- \LaTeXe{}, которая являеться текущей по сей день.

\subsection{Достоинства и недостатки}
Среди достоинств можно выделить:
\begin{itemize}
\item{}Автор может не вникать в детали оформления документа, ему лишь надо
задать логическую структуру текста.
\item{}высокое качество и гибкость верстки абзацев и математических формул.
\item{}\TeX{} не требует большой вычеслительной мощности.
\item{}Система работает на большенства платформах.
\end{itemize}
Среди недостатков можно выделить:
\begin{itemize}
\item{}Исходный текст не будет выглядеть так же как при печати.
\item{}Создание нового макета документа очень трудоемкая задача.
\item{}\TeX{} плохо приспособлен для верстки страниц со сложным взаимодействие текста
и графиков.
\end{itemize}

\subsection{Описание выбранного ПО}
Для написания отчета по практике был выбра ряд программного обеспечения:
\begin{enumerate}
\item Сборка \TeX'а MacTeX(http://tug.org/mactex/), включающий pdfLaTeX, который
выдает документ с расширение pdf.
\item IDE Eclipse(https://www.eclipse.org/) с расширением
TeXlipse(http://texlipse.sourceforge.net/), позволяющие удобно редактировать
документ.
\end{enumerate}
\end{document}
 
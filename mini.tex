\documentclass[14pt,a4paper]{article}
%\renewcommand{\baselinestretch}{1.5}
\usepackage[left=2cm,right=2cm,top=2cm,bottom=2cm,bindingoffset=0cm]{geometry}
\linespread{1.3}
\usepackage[T2A]{fontenc} % Поддержка русских букв
\usepackage[utf8]{inputenc} 
\usepackage[english,russian]{babel}  
\bibliographystyle{unsrt}


 
\begin{document}
\section{Задание}
Построить иммитационую модель мобильной WIMAX-сети, с возможность передвижения
со скоростью 100 км/ч без потери связи.

\section{Описание используемого ПО}
Для разработки исользовалось программое обеспечение
NS-3(https://www.nsnam.org/).

\section{Описание модели}
Разработанная модель состоит из четырех узлов:
\begin{itemize}
  \item{} Мобильная станция(NodeContainer ssNodes;).
  \item{} Базовая станция(NodeContainer bsNodes;) и узел ASN-щлюза
  (NodeContainer ASN\_Node;), которые составляют Access Service Network(Сеть
  доступа).
  \item{} узла CSN(NodeContainer CSN\_Node;).
\end{itemize}



\section{Код разработанной модели}

 {\scriptsize \begin{verbatim}
#include "ns3/core-module.h"
#include "ns3/network-module.h"
#include "ns3/applications-module.h"
#include "ns3/mobility-module.h"
#include "ns3/config-store-module.h"
#include "ns3/wimax-module.h"
#include "ns3/csma-module.h"
#include <iostream>
#include "ns3/global-route-manager.h"
#include "ns3/mobility-module.h"
#include "ns3/internet-module.h"
#include "ns3/vector.h"

NS_LOG_COMPONENT_DEFINE ("wimax");

using namespace ns3;

int main (int argc, char *argv[])
{
  int duration = 10;
  NodeContainer ssNodes;//узел мобильной станции
  Ptr<SubscriberStationNetDevice> ss;
  NetDeviceContainer ssDevs;
  Ipv4InterfaceContainer SSinterfaces;
  NodeContainer bsNodes;// узел базовой станции
  Ptr<BaseStationNetDevice> bs;
  NetDeviceContainer bsDevs, bsDevsOne;
  Ipv4InterfaceContainer BSinterfaces;
  Ptr<SimpleOfdmWimaxChannel> channel;
  NodeContainer CSN_Node;//узлы CSN
  NodeContainer ASN_Node;//узлы asn
  Ptr<ConstantPositionMobilityModel> BSPosition;
  Ptr<RandomWaypointMobilityModel> SSPosition;
  Ptr<RandomRectanglePositionAllocator> SSPosAllocator;

  WimaxHelper::SchedulerType scheduler = WimaxHelper::SCHED_TYPE_SIMPLE;

  LogComponentEnable ("UdpEchoClientApplication", LOG_LEVEL_INFO);
  LogComponentEnable ("UdpEchoServerApplication", LOG_LEVEL_INFO);
  LogComponentEnable ("UdpClient", LOG_LEVEL_INFO);
  LogComponentEnable ("UdpServer", LOG_LEVEL_INFO);

  ssNodes.Create (1);
  bsNodes.Create (1);

  CSN_Node.Create (1);
  ASN_Node.Create (1);

  WimaxHelper wimax;

  channel = CreateObject<SimpleOfdmWimaxChannel> ();
  channel->SetPropagationModel (SimpleOfdmWimaxChannel::COST231_PROPAGATION);
  ssDevs = wimax.Install (ssNodes,
                          WimaxHelper::DEVICE_TYPE_SUBSCRIBER_STATION,
                          WimaxHelper::SIMPLE_PHY_TYPE_OFDM,
                          channel,
                          scheduler);
  Ptr<WimaxNetDevice> dev = wimax.Install (bsNodes.Get (0),
                                           WimaxHelper::DEVICE_TYPE_BASE_STATION,
                                           WimaxHelper::SIMPLE_PHY_TYPE_OFDM,
                                           channel,
                                           scheduler);
  bsDevs.Add (dev);

  SSPosition = CreateObject<RandomWaypointMobilityModel> ();
  SSPosAllocator = CreateObject<RandomRectanglePositionAllocator> ();
  Ptr<UniformRandomVariable> xVar = CreateObject<UniformRandomVariable> ();
  xVar->SetAttribute ("Min", DoubleValue (0));
  xVar->SetAttribute ("Max", DoubleValue (1000));
  SSPosAllocator->SetX (xVar);
  Ptr<UniformRandomVariable> yVar = CreateObject<UniformRandomVariable> ();
  yVar->SetAttribute ("Min", DoubleValue (0));
  yVar->SetAttribute ("Max", DoubleValue (0));
  SSPosAllocator->SetY (yVar);
  SSPosition->SetAttribute ("PositionAllocator", PointerValue (SSPosAllocator));
  SSPosition->SetAttribute ("Speed", StringValue ("ns3::ConstantRandomVariable[Constant=30]"));
  SSPosition->SetAttribute ("Pause", StringValue ("ns3::ConstantRandomVariable[Constant=0.01]"));

  ss = ssDevs.Get (0)->GetObject<SubscriberStationNetDevice> ();
  ss->SetModulationType (WimaxPhy::MODULATION_TYPE_QAM16_12);
  ssNodes.Get (0)->AggregateObject (SSPosition);
  bs = bsDevs.Get (0)->GetObject<BaseStationNetDevice> ();
  CsmaHelper csmaASN_BS;
  CsmaHelper csmaCSN_ASN;

  NodeContainer LAN_ASN_BS;

  LAN_ASN_BS.Add (bsNodes.Get (0));

  LAN_ASN_BS.Add (ASN_Node.Get (0));

  csmaASN_BS.SetChannelAttribute ("DataRate", DataRateValue (DataRate (10000000)));
  csmaASN_BS.SetChannelAttribute ("Delay", TimeValue (MilliSeconds (2)));

  NetDeviceContainer LAN_ASN_BS_Devs = csmaASN_BS.Install (LAN_ASN_BS);

  NetDeviceContainer BS_CSMADevs;

  BS_CSMADevs.Add (LAN_ASN_BS_Devs.Get (0));

  NetDeviceContainer ASN_Devs1;
  ASN_Devs1.Add (LAN_ASN_BS_Devs.Get (1));

  NodeContainer LAN_ASN_CSN;
  LAN_ASN_CSN.Add (ASN_Node.Get (0));
  LAN_ASN_CSN.Add (CSN_Node.Get (0));

  csmaCSN_ASN.SetChannelAttribute ("DataRate", DataRateValue (DataRate (10000000)));
  csmaCSN_ASN.SetChannelAttribute ("Delay", TimeValue (MilliSeconds (2)));

  NetDeviceContainer LAN_ASN_CSN_Devs = csmaCSN_ASN.Install (LAN_ASN_CSN);

  NetDeviceContainer CSN_Devs;
  NetDeviceContainer ASN_Devs2;
  ASN_Devs2.Add (LAN_ASN_CSN_Devs.Get (0));
  CSN_Devs.Add (LAN_ASN_CSN_Devs.Get (1));

  MobilityHelper mobility;
  InternetStackHelper stack;
  mobility.Install (bsNodes);
  stack.Install (bsNodes);
  mobility.Install (ssNodes);
  stack.Install (ssNodes);
  stack.Install (CSN_Node);
  stack.Install (ASN_Node);

  Ipv4AddressHelper address;

  address.SetBase ("192.168.1.0", "255.255.255.0");
  bsDevsOne.Add (bs);
  BSinterfaces = address.Assign (bsDevsOne);
  SSinterfaces = address.Assign (ssDevs);

  address.SetBase ("192.168.2.0", "255.255.255.0");
  Ipv4InterfaceContainer BSCSMAInterfaces = address.Assign (BS_CSMADevs);
  Ipv4InterfaceContainer ASNCSMAInterfaces1 = address.Assign (ASN_Devs1);

  address.SetBase ("192.168.3.0", "255.255.255.0");
  Ipv4InterfaceContainer ASNCSMAInterfaces2 = address.Assign (ASN_Devs2);
  Ipv4InterfaceContainer CSNCSMAInterfaces = address.Assign (CSN_Devs);

  Ipv4Address multicastGroup ("224.30.10.81");

  Ipv4StaticRoutingHelper multicast;
  
  Ptr<Node> multicastRouter = ASN_Node.Get (0); 
  Ptr<NetDevice> inputIf = ASN_Devs2.Get (0); 

  multicast.AddMulticastRoute (multicastRouter, CSNCSMAInterfaces.GetAddress(0), multicastGroup, 
  inputIf, ASN_Devs1);

  Ptr<Node> sender = CSN_Node.Get (0);
  Ptr<NetDevice> senderIf = CSN_Devs.Get (0);
  multicast.SetDefaultMulticastRoute (sender, senderIf);

  multicastRouter = bsNodes.Get (0); 
  inputIf = BS_CSMADevs.Get (0); 

  multicast.AddMulticastRoute (multicastRouter, SSinterfaces.GetAddress(0), multicastGroup, 
  inputIf, bsDevsOne);

  Ipv4StaticRoutingHelper multicast1;
  
  Ptr<Node> multicastRouter1 = bsNodes.Get (0); 
  Ptr<NetDevice> inputIf1 = bsDevsOne.Get (0); 

  multicast1.AddMulticastRoute (multicastRouter1, SSinterfaces.GetAddress(0), multicastGroup, 
  inputIf1, BS_CSMADevs);

  Ptr<Node> sender1 = ssNodes.Get (0);
  Ptr<NetDevice> senderIf1 = ssDevs.Get (0);
  multicast1.SetDefaultMulticastRoute (sender1, senderIf1);

  multicastRouter1 = ASN_Node.Get (0); 
  inputIf1 = ASN_Devs1.Get (0); 

  multicast1.AddMulticastRoute (multicastRouter1, CSNCSMAInterfaces.GetAddress(0), multicastGroup, 
  inputIf1, ASN_Devs2);

  // настройка приложений на сервере
  UdpServerHelper udpServer;
  ApplicationContainer serverApps;
  UdpClientHelper udpClient;
  ApplicationContainer clientApps;

  udpServer = UdpServerHelper (100);

  serverApps = udpServer.Install (CSN_Node.Get (0));
  serverApps.Start (Seconds (1));
  serverApps.Stop (Seconds (duration));

  udpClient = UdpClientHelper (multicastGroup, 100);
  udpClient.SetAttribute ("MaxPackets", UintegerValue (3));
  udpClient.SetAttribute ("Interval", TimeValue (Seconds (0.5)));
  udpClient.SetAttribute ("PacketSize", UintegerValue (1024));

  clientApps = udpClient.Install (ssNodes.Get (0));
  clientApps.Start (Seconds (3));
  clientApps.Stop (Seconds (duration));

  UdpServerHelper udpServer1;
  ApplicationContainer serverApps1;
  UdpClientHelper udpClient1;
  ApplicationContainer clientApps1;

  udpServer1 = UdpServerHelper (99);

  serverApps1 = udpServer1.Install (ssNodes.Get (0));
  serverApps1.Start (Seconds (1));
  serverApps1.Stop (Seconds (duration));

  udpClient1 = UdpClientHelper (multicastGroup, 99);
  udpClient1.SetAttribute ("MaxPackets", UintegerValue (3));
  udpClient1.SetAttribute ("Interval", TimeValue (Seconds (0.5)));
  udpClient1.SetAttribute ("PacketSize", UintegerValue (1024));

  clientApps1 = udpClient1.Install (CSN_Node.Get (0));
  clientApps1.Start (Seconds (5));
  clientApps1.Stop (Seconds (duration));

  IpcsClassifierRecord MulticastClassifier (BSinterfaces.GetAddress(0),
                                            Ipv4Mask ("255.255.255.255"),
                                            SSinterfaces.GetAddress(0),
                                            Ipv4Mask ("255.255.255.255"),
                                            101,
                                            65000,
                                            0,
                                            100,
                                            17,
                                            1);

  ServiceFlow MulticastServiceFlow = wimax.CreateServiceFlow (ServiceFlow::SF_DIRECTION_DOWN,
                                                              ServiceFlow::SF_TYPE_UGS,
                                                              MulticastClassifier);

  ss->AddServiceFlow (MulticastServiceFlow);

  bsNodes.Get(0)->GetObject<MobilityModel>()->SetPosition(Vector(0,0,0));
  ssNodes.Get(0)->GetObject<MobilityModel>()->SetPosition(Vector(100,0,0));
  
  Simulator::Stop (Seconds (duration));
  Simulator::Run ();

  Ptr<MobilityModel> mob = ssNodes.Get(0)->GetObject<MobilityModel>();
  Vector pos = mob->GetPosition ();
  std::cout << "POS: x=" << pos.x << ", y=" << pos.y << std::endl;

  Simulator::Destroy ();
  return 0;
}

 \end{verbatim}}

\end{document}